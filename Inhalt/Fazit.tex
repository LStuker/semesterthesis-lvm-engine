\chapter{Fazit / Zusammenfassung}
\label{cha:Fazit}

Mit dem Thema Volume Manager Engine habe ich bewusst ein Thema gewählt, welches mir erlaubt, viel neues Wissen anzueignen und mich in neue Technologien einzuarbeiten. Neben den Volume Managern LVM und ZFS war mir die Programmiersprache Ruby zu Beginn der Arbeit nicht bekannt.

Für die Einarbeitung in LVM habe ich mich zuerst mittels Literatur in das Thema eingelesen. Dazu habe ich das Handbuch von RedHat verwendet. Für eine wissenschaftliche Arbeit möchte man sicher noch tiefer in eine neue Technologie hineinblicken, als dies das Studium eines Benutzerhandbuches für Anwender zulässt. Leider war es mir trotz intensiven Recherchen nicht möglich gewesen, an bessere und zuverlässigere Quellen heranzukommen. Beispielsweise ging aus dem Handbuch nicht klar hervor, wie die Segmentierung der Volumes und das Striping der Volumens zusammenhängen. Leider blieb eine anonyme Anfrage in der offiziellen LVM-Mailingliste unbeantwortet und erfolglos.
Andere Fragen versuchte ich anhand von virtuellen Linux Servern, welche ich als Cluster und nicht Cluster-Betrieb ausführte, mir zu erarbeiten. Obwohl diese Vorgehensweise eher aufwendig ist, sammelte ich damit viel Erfahrung und es half mir sehr gut, mich in LVM einzuarbeiten.
 
Anhand von Grafiken und den dazugehörenden Beschreibungen, versuchte ich die Eigenschaften von LVM dem Leser zu erklären. Die Grafiken sollten es dem Leser vereinfachen, die verschiedenen Volume Architekturen, welche LVM unterstützt und wie die einzelnen LVM-Objekte zu einander in Verbindung stehen, zu verstehen. Die Zusammenfassung der wichtigsten Befehle beschreibt, wie sich die aufgeführten Architekturen konfigurieren lassen.
 
Die Anforderungsanalyse für die Applikation habe ich um eine Marktanalyse ergänzt, obwohl diese nicht Ziel der Arbeit war, aber weitere sinnvolle Informationen liefert. Eine Marktanalyse sollte nach meinem Verständnis in keiner Arbeit fehlen, in der es um die Implementation eines neuen Produktes geht. Ohne zu erforschen, welche Produkte und Optionen der Markt für die bestehenden und zukünftigen Technologien bietet und welche davon sich wahrscheinlich durchsetzen werden, sollte man kein Softwareprojekt beginnen. Gerne hätte ich anhand von Gartner-Analysen dargestellt, wie sich der Markt von x86 zu RISC Architekturen voraussichtlich künftig entwickeln wird, jedoch waren diese Analysen nur käuflich zu erwerben und nicht öffentlich zugänglich. Eine Firma hätte sicherlich sich die notwendigen Informationen aus dem Projektbudget geleistet. 
  
Mit dem Prototyp konnte ich aufzeigen, dass eine Umsetzung des von mir erstellten Konzept möglich ist. Für den Prototyp habe ich ein bestehendes Projekt verwendet und mit meinen Anforderungen erweitert. Als Novize in einer neuen Programmiersprache hat es mir geholfen, dass ich mich an bestehenden Codebeispielen orientiert habe und meiner Einschätzung nach eine akurate Architektur erstellt habe.

Für mich persönlich fällt die Zwischenbilanz für diese Arbeit positiv aus. Die Arbeit half mir neues und für meinen Beruf notwendiges Wissen anzueignen, welches ich künftig in meinen Arbeiten einsetzen werde. Auf Wunsch meines Arbeitgebers werde ich mich nach Beendung der Semesterarbeit bei der Entwicklung neuer Storage Managment Tools in der Firma einbringen.
Dieser neue Aufgabenbereich bietet mir neben der Weiterbildung im System-Engineering, einen nahtlosen und idealen Einstieg in die Softwareentwicklung.




